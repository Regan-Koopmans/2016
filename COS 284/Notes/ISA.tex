% Created 2016-10-09 Sun 09:51
\documentclass[11pt]{article}
\usepackage[utf8]{inputenc}
\usepackage[T1]{fontenc}
\usepackage{fixltx2e}
\usepackage{graphicx}
\usepackage{grffile}
\usepackage{longtable}
\usepackage{wrapfig}
\usepackage{rotating}
\usepackage[normalem]{ulem}
\usepackage{amsmath}
\usepackage{textcomp}
\usepackage{amssymb}
\usepackage{capt-of}
\usepackage{hyperref}
\author{Regan}
\date{\today}
\title{Fundamental Concepts and ISA}
\hypersetup{
 pdfauthor={Regan},
 pdftitle={Fundamental Concepts and ISA},
 pdfkeywords={},
 pdfsubject={},
 pdfcreator={Emacs 25.1.1 (Org mode 8.3.6)}, 
 pdflang={English}}
\begin{document}

\maketitle
\tableofcontents



\section{Why Study Computer Architecture?}
\label{sec:orgheadline2}

\begin{itemize}
\item Learn about how computers work and how they are designed
\item Creating computing systems that satify goals.
\item How have these goals changed?
\end{itemize}

It is okay not to know something, you don't know most things.
Read \textbf{Gordon Moore's} seminal paper.

\subsection{Today is a critical time for computer architecture}
\label{sec:orgheadline1}

\textbf{20 years ago there was one architecture that most people agreed on.}

This is not the case of today.

There is a massive paradigm shift to multicorea processors, there
is currently no optimal solutions. Dynamically reconfigurable 
architecures.

Out-order execution allow for massively parallized executions,
accross multiple cores. "Paging" instructions. Power and energy
has become increasingly more important (because of mobile devices).

Smaller DRAM requires more frequent updates. Take advantage of user
level perception and requirements. 

Computation becomes faster, and this effectively makes the cost of 
access to memory higher, since this speed generally stays constent.
Memory is becoming a bigger cost.

Not every problem is parallizable (or is it?). Implicit destinations.
The structure of scientific revolutions. Pre-paradigm

\section{Understanding Fundamentals}
\label{sec:orgheadline7}

"\emph{One can only change the world once one understands it.}"

A computer system is fundamentally made of the following three
components.

\uline{Computation}

How do we calculate things?

\uline{Communication}

How do subcomponents talk to one another and collaborate?

\uline{Storage (memory)}

How do we make short term and long term persistence happen?

\subsection{The von Neumann Computer Architecture}
\label{sec:orgheadline3}

\begin{itemize}
\item Stored program in memory
\begin{itemize}
\item Linear array in terms of memory.
\item Interpretation of stored value depends on context in cycle.
\end{itemize}

\item Sequential instruction processing
\begin{itemize}
\item only one instruction is processed at a time, and you keep 
going. "Instruction pointer".
\end{itemize}
\end{itemize}

The instruction processing cycle.
\begin{itemize}
\item Fetch
\item Decode
\item Evaluate operands
\item Fetch operands
\item Execute
\item Store (result)
\end{itemize}

\subsection{The Dataflow Model}
\label{sec:orgheadline5}

Instructions execute when operands are loaded, rather than when
the program counter points to it. No program counter.

Processor keeps track of dependencies. There is no sequential
constraint, since dependencies may be one to many and many to
one. Read more about this. Data flow graphs. Is a pull model.

Are humans sequential? Von Neumann offers precise state.

\subsubsection{Barrier Synchronization}
\label{sec:orgheadline4}

Different threads of control that we want to synchronise at
some point.

Say we have a book, we want to count the number of certain
letters in the book. You can have a separate thread for each
letter. Alternatively you can allocate a page to each thread.

We want some \textbf{massive synchronization point}. The dataflow 
model exposes more parallels, but very complex to design.
The problem is maintaining states.

\texttt{A preliminary architecture for a basic data-flow processor.}

\subsection{ISA vs Microarchitecture}
\label{sec:orgheadline6}

A car has an interface to the driver.
\begin{itemize}
\item Pedal accelerates
\item Wheel steers.
\end{itemize}

As drivers we do not know the difference between cars. Think
of add instruction, versus the implementation of a full adder.
Bit serial adders.

We have tradeoffs even at the arithmetic level.
\end{document}
