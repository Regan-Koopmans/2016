\documentclass[10pt,a4paper]{article}
\usepackage[latin1]{inputenc}
\usepackage{amsmath}
\usepackage{amsfonts}
\usepackage{amssymb}
\usepackage{graphicx}
\usepackage{listings}



\title{Benchmarking Memory and Computational Efficiency of Various Languages}

\author{COS 284, Regan Koopmans}

\begin{document}
	\maketitle
	
	\section*{Testing Methods}
	
		In order to test the time taken when running a program, I used the '\texttt{time}' bash command. This command has a 1 millisecond accuracy. In order to record the averages and control the test sequences, I constructed the following short bash script:
	
	\hrulefill
	
		\begin{lstlisting}
		
 #!/bin/bash
 for i in {1..100}
   do
     var=$(./getTime.sh 2>&1)
     string="$var"
     for i in {1..999}
       do
         var=$(./getTime.sh 2>&1)
         string="$string + $var"
       done
     echo $(python -c "print ($string)/1000")
 done
		
		\end{lstlisting}

	\hrulefill
	\\
		In this case \texttt{getTime.sh} is a small script that runs and measures a program, dependent on its specific running requirements (such as invoking Java or Lisp). The results were then sorted and stored in a file for later comparison.
	
	
	
	\section*{Results}
		\subsection*{Memory}

		\begin{center}
		\bgroup
		\def\arraystretch{1.5}
			\begin{tabular}{|l|l|}
				\hline
				\textit{Language} & \textit{Space Occupied on Disk}	\\ \hline
				Assembly & 4 KB                          			\\
				C++      & 12 KB                         			\\
				COBOL    & 16 KB                         			\\
				Fortran  & 12 KB                         			\\
				Lisp     & 4 KB (clisp binary is 9.5 MB) 			\\
				Java     & 4 KB (JVM is approx. 150 MB)  			\\ \hline
			\end{tabular}
		\egroup
		\end{center}
		
		\subsection*{Computation Time}
	
	\section*{Conclusion}
	
	It is obvious that, which concurs with expectation. This is however only a limited test, and languages may be faster or slower than one another in certain environments and tasks.
	
\end{document}