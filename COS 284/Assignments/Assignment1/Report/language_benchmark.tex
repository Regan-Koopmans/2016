\documentclass[11pt,a4paper]{article}
\usepackage[latin1]{inputenc}
\usepackage{amsmath}
\usepackage{amsfonts}
\usepackage{amssymb}
\usepackage{graphicx}
\usepackage{listings}



\title{Benchmarking Memory and Computational Efficiency of Various Languages}

\author{COS 284, Regan Koopmans}

\begin{document}
	\maketitle
	
	\section*{Testing Methods}
	
		In order to test the time taken when running a program, I used the '\texttt{perf}' bash command, which can be used for a variety of performance testing applications. This was used over the \texttt{time} command, which proved to be too inaccurate for valuable results. In order to record the averages and control the test sequences, I constructed the following short bash script:
	
	\hrulefill
	
		\begin{lstlisting}
		
  #!/bin/bash
  for i in {1..100}
  do
    echo $(perf stat -r 1000 ./HelloWorld 2>&1 >/dev/null 
    | tail -n 2 | sed 's/ \+//' | sed 's/ /,/' 
    | sed 's/[^0-9.]//g')
  done

		
		\end{lstlisting}

	\hrulefill
	\\
	\\	This was run on each program, dependent on its specific running requirements (such as invoking Java or Lisp). The results were then sorted and stored in a text file for later comparison. An example of the output can be seen below (results are recorded in seconds) :
	
			\begin{lstlisting}
			
		0.0007464720
		0.0007465560
		0.0007470370
		0.0007487890
		0.0007489340
		...
			
			
		\end{lstlisting}

	\section*{Results}
		\subsection*{Memory}
		
		\begin{center}
		\bgroup
		\def\arraystretch{1.5}
			\begin{tabular}{|l|l|}
				\hline
				\textit{Language} & \textit{Space Occupied on Disk}	\\ \hline
				Assembly & 4 KB                          			\\
				C++      & 12 KB                         			\\
				COBOL    & 16 KB                         			\\
				Fortran  & 12 KB                         			\\
				Lisp     & 4 KB (clisp binary is 9.5 MB) 			\\
				Java     & 4 KB (JVM is approx. 150 MB)  			\\ \hline
			\end{tabular}
		\egroup
		\end{center}
		\space
		\subsection*{Computation Time}
			\space
			\begin{center}
				\bgroup
				\def\arraystretch{1.5}
				\begin{tabular}{|l|l|}
					\hline
					\textit{Language} & \textit{Best Average Run Time} \\ \hline
					Assembly & 0.1489200 ms          \\
					C++      & 0.8101980 ms          \\
					COBOL    & 0.7464720 ms          \\
					Fortran  & 0.5536200 ms          \\
					Lisp     & 6.9800960 ms          \\
					Java     & 48.8928690 ms                     \\ \hline
				\end{tabular}
				\egroup
			\end{center}
	\section*{Conclusion}
	
	It is obvious that the Assembly implementation is the most proficient in both excecution time and binary size, which concurs with expectation. The compiled languages all complete within less than a millisecond, and the interpreted languages run dramatically slower (almost 50 times slower in Java's case). This is however only a limited test, and languages may be faster or slower than one another in certain environments and tasks.
	
\end{document}